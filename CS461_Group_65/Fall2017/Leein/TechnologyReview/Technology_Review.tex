\documentclass[onecolumn, draftclsnofoot,10pt, compsoc]{IEEEtran}
\usepackage{graphicx}
\usepackage{url}
\usepackage{setspace}

\usepackage{geometry}
\geometry{textheight=9.5in, textwidth=7in}

% 1. Fill in these details
\def \CapstoneTeamName{		The Cleverly Named Team}
\def \CapstoneTeamNumber{		65}
\def \GroupMemberOne{			Inhyuk Lee}
\def \CapstoneProjectName{			CDK Data Stream AI}
\def \CapstoneSponsorCompany{	CDK Global}
\def \CapstoneSponsorPerson{		Chris Smith}

% 2. Uncomment the appropriate line below so that the document type works
\def \DocType{		Problem Statement
				%Requirements Document
				%Technology Review
				%Design Document
				%Progress Report
				}
			
\newcommand{\NameSigPair}[1]{\par
\makebox[2.75in][r]{#1} \hfil 	\makebox[3.25in]{\makebox[2.25in]{\hrulefill} \hfill		\makebox[.75in]{\hrulefill}}
\par\vspace{-12pt} \textit{\tiny\noindent
\makebox[2.75in]{} \hfil		\makebox[3.25in]{\makebox[2.25in][r]{Signature} \hfill	\makebox[.75in][r]{Date}}}}
% 3. If the document is not to be signed, uncomment the RENEWcommand below
%\renewcommand{\NameSigPair}[1]{#1}

%%%%%%%%%%%%%%%%%%%%%%%%%%%%%%%%%%%%%%%
\begin{document}
\begin{titlepage}
    \pagenumbering{gobble}
    \begin{singlespace}
    	\includegraphics[height=4cm]{coe_v_spot1}
        \hfill 
        % 4. If you have a logo, use this includegraphics command to put it on the coversheet.
        %\includegraphics[height=4cm]{CompanyLogo}   
        \par\vspace{.2in}
        \centering
        \scshape{
            \huge CS Capstone \DocType \par
            {\large\today}\par
            \vspace{.5in}
            \textbf{\Huge\CapstoneProjectName}\par
            \vfill
            {\large Prepared for}\par
            \Huge \CapstoneSponsorCompany\par
            \vspace{5pt}
            {\Large\NameSigPair{\CapstoneSponsorPerson}\par}
            {\large Prepared by }\par
            Group\CapstoneTeamNumber\par
            % 5. comment out the line below this one if you do not wish to name your team
            \CapstoneTeamName\par 
            \vspace{5pt}
            {\Large
                \NameSigPair{\GroupMemberOne}\par
            }
            \vspace{20pt}
        }
        \begin{abstract}
        % 6. Fill in your abstract    
        	On this document, potential three problems on capstone project will be discussed, and nine technologies are compared and contrast as a potential solution. On each problem, one technology is selected as a solution.
        \end{abstract}     
    \end{singlespace}
\end{titlepage}
\newpage
\pagenumbering{arabic}
\tableofcontents
% 7. uncomment this (if applicable). Consider adding a page break.
%\listoffigures
%\listoftables
\clearpage

% 8. now you write!
\section{Overview:}
On this section, most commonly used deep learning library will be compare. The three library that will be compared are TensorFlow, Deep Learning4 Java(DL4J), and OpenCV. The best library that fit into our project need will be selected.

\section{Criteria:}
Our main goal of the project is to build the signature recognition program by using image detecting API. To accomplish this task, it is essential to find suitable machine learning library for signature recognition.  Machine learning library must be runnable in different operating system, support many programming language, provide handwriting recognition algorithms or functions, and fast enough. 

\section{Potential Choices:}
\subsection{TensorFlow}
 Advantages: TensorFlow is a machine learning library released in November 9. It has short history. However, large community use TensorFlow, and there are open source repositories more than 6000 on online. TensorFlow has special feature on numerical computation makes TensorFlow as powerful tool. Unlike other machine learning library, TensorFlow uses data flow graphs that contains vectors or matrixes called tensor, and this data flow graphs allow developers to compute the calculation on one or more CPUs or GPUs.
The other strength of TensorFlow is and flexibility. It supports both laptop/desktop operating system and mobile operating system such as 64-bit Linux, MacOS, Windows, Android, and IOS. Also, TensorFlow application program interface (API) supports in Python, Java, C, and Go.
It supports Modified National Institute of Standards and Technology (MNIST) database. MNIST is the database that has images of handwritten digits that is commonly used for training image processing system. It contains about 60,000 set of training examples, and 10,000 set of test examples. TensorFlow provides tutorial for machine learning beginner.

Disadvantages: There are also disadvantages using TensorFlow. The main problem with TensorFlow is the speed. It has dramatically slower speed compare of other frameworks. Also, TensorFlow is heavier than other machine learning libraries. Since TensorFlow force to use set of vector or matrix for flow graph data, copying large matrices cause slowness of the program. The last problem of the TensorFlow is the optimization of Java API. It is experimental step supporting Java API, so Java API for TensorFlow is not stable.

\subsection{Deep Learning for Java(DL4J)}
 Advantages: Deep Learning for Java (DL4J) is the machine learning library supports on Java virtual machine (JVM). It supports languages such as Java, Scala, Clojure, and Kotlin, but Java is the main language. The strength of the DL4J is the speed. There are several characteristics that makes DL4J run faster. First characteristic is the DL4J has linear algebra computations. It is much simpler than other complicated computation algorithm. The second characteristic is that DL4J can run in parallel, and this parallel setting is automatic. Since user does not have to create parallel setting, system runs much faster. The last characteristic is that DL4J relies on JavaCPP which parse C/C++ header files to Java interface files. This allows DL4J run faster since most of the high cost program are written in C or C++.
Although DL4J API language support lack of diversity, it won’t cause much problem since Java is one of the largest programming language. Almost 10 million developers use this language, and many large companies highly rely on Java or a JVM based system. In addition, Java supports in many platforms without converting the language. Once developers write the code, they simply have to pass their code in JVM to run on different operating system.
The last advantage of DL4J is that it can import neural net models from TensorFlow, and it also supports MNIST database like TensorFlow.

Disadvantages: There are two main disadvantages using this machine learning library. First one is that DL4J does not support many languages. Although Java can support different environment by using JVM, different programming languages has strength on different circumstances. Supporting more programming language will to give diversity to developers. The second problem is that there are few community compare to other machine learning library because DL4J has short history.

\subsection{OpenCV}
 Advantages: Unlike previous two machine learning libraries, OpenCV has long history. It is originally released on June 2000. Since OpenCV has a long history, it has large community. According to OpenCV website, there are 47 thousand people of user community, and more than 14 million number of downloads happened. Additionally, our teaching assistance have experience on OpenCV. This is highly advantageous since our group can get advice directly from our TA.
Due to the long history of the OpenCV, it provides more than 2500 optimized algorithms. These algorithms can be used to implement many images related deep learning programs. It also supports MNIST, and have tutorial on it.
OpenCV is more flexible then TensorFlow. It runs on both desktop (Windows, Linux, Android, MacOS, FreeBSD, OpenBSD) and mobile (Android, Maemo, iOS), and supports C++, C, Python and Java languages.
Fastness is another benefit of using OpenCV. Since OpenCV library is written in C/C++, it is fast. Also, OpenCV currently supports CUDA and OpenCL about 500 algorithms to accelerate the original algorithms. CUDA and OpenCL are the modern GPU accelerators. These GPU accelerators increase GPU performance, so computation time for algorithms shrinks.


\section{Discussion, compare and contrast:}
As it mentioned on criteria section, machine learning library needs to support enough algorithms, language, and operating system. Also, it need to have fast computation speed, and large community. Following table shows the compare and contrast based on these requirements.

\begin{table}[h]
\centering
\begin{tabular}{ | c | c | c | c |}
\hline
 & TensorFlow & DL4J & OpenCV\\ 
\hline
handwriting recognition algorithms & Have decent algorithms & Have decent algorithms & Have decent algorithms \\ 
\hline
Language support & Many & very few & Many\\ 
\hline
Operating system support & Enough & Enough & Many\\ 
\hline
Community & Many & Few & Abound\\ 
\hline  
Speed & Slow & Fast & Fast\\ 
\hline  
\end{tabular}
\caption{Compare and Contrast}
\label{table:1}
\end{table}

\section{Conclusion:}
From the discussion, compare and contrast section, it is clear that OpenCV is the best choice for CDK Data Stream AI project. It has enough handwriting recognition algorithm, supports various language and operating system, has large community, and fast speed.
\end{document}