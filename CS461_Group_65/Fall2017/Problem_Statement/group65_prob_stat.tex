\documentclass[onecolumn, draftclsnofoot,10pt, compsoc]{IEEEtran}
\usepackage{graphicx}
\usepackage{url}
\usepackage{setspace}

\usepackage{geometry}
\geometry{textheight=9.5in, textwidth=7in}

% 1. Fill in these details
\def \CapstoneTeamName{		GROUP65}
\def \CapstoneTeamNumber{		65}
\def \GroupMemberOne{			Jacob Geddings}
\def \GroupMemberTwo{			Inhyuk Lee}
\def \GroupMemberThree{			Juan Mugica}
\def \CapstoneProjectName{		CDK Data Stream AI}
\def \CapstoneSponsorCompany{	CDK Global}
\def \CapstoneSponsorPerson{		Chris Smith}

% 2. Uncomment the appropriate line below so that the document type works
\def \DocType{		Problem Statement
				%Requirements Document
				%Technology Review
				%Design Document
				%Progress Report
				}
			
\newcommand{\NameSigPair}[1]{\par
\makebox[2.75in][r]{#1} \hfil 	\makebox[3.25in]{\makebox[2.25in]{\hrulefill} \hfill		\makebox[.75in]{\hrulefill}}
\par\vspace{-12pt} \textit{\tiny\noindent
\makebox[2.75in]{} \hfil		\makebox[3.25in]{\makebox[2.25in][r]{Signature} \hfill	\makebox[.75in][r]{Date}}}}
% 3. If the document is not to be signed, uncomment the RENEWcommand below
\renewcommand{\NameSigPair}[1]{#1}

%%%%%%%%%%%%%%%%%%%%%%%%%%%%%%%%%%%%%%%
\begin{document}
\begin{titlepage}
    \pagenumbering{gobble}
    \begin{singlespace}
    	\includegraphics[height=4cm]{coe_v_spot1}
        \hfill 
        % 4. If you have a logo, use this includegraphics command to put it on the coversheet.
        %\includegraphics[height=4cm]{CompanyLogo}   
        \par\vspace{.2in}
        \centering
        \scshape{
            \huge CS Capstone \DocType \par
            {\large\today}\par
            \vspace{.5in}
            \textbf{\Huge\CapstoneProjectName}\par
            \vfill
            {\large Prepared for}\par
            \Huge \CapstoneSponsorCompany\par
            \vspace{5pt}
            {\Large\NameSigPair{\CapstoneSponsorPerson}\par}
            {\large Prepared by }\par
            Group\CapstoneTeamNumber\par
            % 5. comment out the line below this one if you do not wish to name your team
            %\CapstoneTeamName\par 
            \vspace{5pt}
            {\Large
                \NameSigPair{\GroupMemberOne}\par
                \NameSigPair{\GroupMemberTwo}\par
                \NameSigPair{\GroupMemberThree}\par
            }
            \vspace{20pt}
        }
        \begin{abstract}
        % 6. Fill in your abstract    
		Our team has been assigned to assist in the development of AI for application to CDK’s existing Data Streams. 
		The goal in doing this is to gain insight from said data streams, use this data to predict future events, and detect when an anomaly is present. 
		These three goals need to be independent of one another and be capable of functioning as a black box. 
		This entails the ability to write applications and/or functions that are then fed through the system in various ways. 
		Tools that will be necessary for project completion include the AWS platform, Docker, Linux platform, and several open source AI resources.	
        \end{abstract}     
    \end{singlespace}
\end{titlepage}
\newpage
\pagenumbering{arabic}
\tableofcontents
% 7. uncomment this (if applicable). Consider adding a page break.
%\listoffigures
%\listoftables
\clearpage

% 8. now you write!
\section{Defining the Problem}
	
	CDK currently operates with thousands of different dealerships across the world each of which using their own unique forms, servers, and verification methods for each transaction. This current system has allowed a great level of liberty for each dealership but has made it a logistical nightmare to quickly coordinate documents and to ensure everything submitted is actually valid. Their main concern is finding a way to quickly verify if a form has been signed and that a given driver’s license is valid for the region. Beyond that core issue they are also interested in reigning in their various servers so that this verification process is done under one roof and with a singular database. Once achieved they have additional stretch challenges that include the desire to extensively check if a license is valid, ensure that data being checked is kept secure, and that forms are correctly categorized after being checked. On top of these they are interested in more challenging obstacles such as constructing consumer portfolios based on information provided, vehicle processing, vehicle categorization, and mood testing of documents.
	
\section{Proposed Solution}

	The proposed solution is to incorporate AI as a form of black box that all data must pass through. This AI will need to be versatile enough that it can handle multiple forms of data passing through it. As the data is fed through, it should be able to document said data and reference it against other data that has already passed through. In doing this, the AI must be able to determine likely future scenarios given past events. It must also detect when something is drastically outside of what is expected. This will be accomplished through the use of open source AI with a focus on image processing like TensorFlow, Clarifai, and Caffe. In addition to this there is promising options within analytics and business such as H20, Caffe, Deeplearning4j, and Apache SystemML. This will also require a platform off of which to operate, such as Docker, which enables a universal platform via a program that can be installed on multiple machines with minimal system conflicts. An environment to work on this program can also be provided through AWS services, which allow access to various integrated development environments and software development kits all under one roof. Another convenient functionality of AWS is the ability to utilize serverless developer tools such as AWS Serverless Application Model that natively supports CloudFormation. To also meet the need for the program to be portable, we will be utilizing Docker as a general platform. This will give us an easy-to-use container in which to store our program that will ideally result in minimal conflict with whatever system on which it is installed. For security and stability concerns, this will be written for *nix based platforms with a focus on CentOS, which provides a consistent, compatible, and reliable platform. Lastly, as a cloud based platform we’ll need to ensure it can properly communicate with multiple devices simultaneously without causing errors or dropping connections. It will also need to be tested for its ability to reliably communicate with wireless devices in the event CDK seeks to expand this design to accomplish more tasks. The language itself will likely be a variant of Java with AWS providing the cloud network link for this project. Given the scope of this project, we will need to maintain close contact with our client to ensure it remains on track with the desired end goal with communication on a weekly basis. This will mean three key building blocks needing to be completed that will then be linked to create a uniform design: the AI will need to be developed to properly handle the data it is fed; the container in which the program will reside must be constructed; and the network handling must be robust enough to handle multiple connections.
	
\section{Performance Metrics}
	
	There are a few hard requirements and several stretch requirements in regards to the methods of gauging project completeness. Hard requirements include the following. Each functionality must be independent such that analytics and error detection are not bundled together but can instead only use one or the other. This must follow a black box design which does not require the user of the end product to know the inner workings of the program; the user needs to only know what the expected input and output. The system needs to work across several different machines. Lastly, the AI must correctly identify at least if a provided form is signed. As for stretch requirements, the program will need to be capable of returning confirmation if a license is valid. It will need to identify if a vehicle image and name match correctly. It must also be capable of running for extended periods of time without failure. It must correctly communicate with multiple different inputs without collision or hang-ups. It needs to be optimized to a sufficient level that the performance impact on the existing structure is kept to a minimum. This will also require an easy method of maintenance or training for the program when requiring operator corrections should it give false positives or negatives. Anomaly detection will need to be extremely robust and capable of dealing with potential small errors in a time efficient manner. Documentation of what went into the program and how it functions will need to be suitably detailed such that a new team can take over. In the event that the endeavor proves incapable of being completed, a detailed report will be required as to the what, where, why, and when said failure occurred. In either event, documentation must explicitly indicate every source used in the creation of the project with how it was used and where. During the development process, the success of the project can be gauged by several means. One such method will be through contact with CDK: should the project not go in a direction they want, it can be easily corrected through the established email contact method with our client. Communicating with Oregon State University will also help maintain a measure of success. This will include speaking with our teaching assistant and our instructors for guidance and instruction on these various new areas of study. Lastly, as the project begins to near completion, a traditional burn down chart should assist in determining what is left to do and in what areas.
	

\end{document}